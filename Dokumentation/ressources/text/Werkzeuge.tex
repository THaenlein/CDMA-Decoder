Entwickelt wurde unter Windows 10 mit Visual Studio und MSVC, wobei das Projekt auf eine CMake-Umgebung aufsetzt. Anschließend wurde noch unter Fedora mit GCC getestet. Dabei umfasst eine mitgelieferte Makefile die Build-Konfiguration des Release-Builds für GCC. Alternativ lässt sich mit CMake eine Konfiguration für die eigene Plattform selbst generieren.
Zur Leistungsanalyse wurde die CPU-Profilerstellung des Visual-Studio-Diagnosetools aktiviert. So konnten die CPU-Zeit intensivsten Programmpfade ausfindig gemacht werden, um gezielt diesen Code zu verbessern.